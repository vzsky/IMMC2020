\ \ \ \ In order to use our mathematical model, we need to accept some assumptions including..
\begin{enumerate}
    \item Time wasted when the shoppers picking-up the products and when they are at the cashier are vanishingly small and can be neglected.
    \item There is only small amount of area which the customer have to be in, in order to be able to access the products. In reality, only finite number of people can be in that area. We assume that this area can be filled with infinite amount of people. So we assume this area to be the only area that are used to calculate the chaos(level of the damage of the product). This area is used to calculate the density of the customer who want that particular product. This area is called effective area( $0.955 m^2$)
    \item Every products has the same durability, so the damage dealt to the products only depends on their price and size.
    \item Damage is monotonically increased with respect to chaos. This allow us to construct the model that describe the chaos of the shop instead of the damage.
    \item Demand and price are linearly correlated. When the price is decrease, the demand will be increase. So the slope of the price-demand graph will be negative(both $Ed_{producttype},Cd_{product}$ are negative).
    \item Products abide by elastic economic model and elastic change of demand of products in the same type($Ed_{product type}$) are equal.
    \item The amount of products in flash sale is equal to the amount which are sold on non-promotion day, so all the products are expected to be sold out.
    \item Products rating, regardless of type, have the same effect to the demand of the products.
    \item Only the products on the side of the shelf can be accessed by the shopper so we can calculate damage by using only the perimeter of the shelf.
    \item All customers know about the layout beforehand. We can assume that the customers will spread overall the layout equally.
\end{enumerate}
 