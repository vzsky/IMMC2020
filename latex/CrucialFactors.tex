

\ \ \ \ The popularity of the product indicates the competitiveness, the rivalry of that products, the messiness, and the chaos which all resulted as damage. We need to consider all the factors that affect the popularity of the products and what property of can attract the customers. 
\newline

Firstly, we consider what will attract the shopper or what factors will affect the popularity of the product. Price and discount is the first one we need to consider. The lower the price the more the shoppers want to buy. The higher the discount the more they want to get this product. These two factors can be gathered from the given list. Brand and brand loyalty somehow play a big role in people decision. But we can't get this information so we considered it to be the same for all the products. Have you ever heard of the law of demand and supply? The less the products, the more rivalry there is for the customers Hence, the more the shoppers the higher the competitive tendency.
\newline

Secondly, to analyse damage value, we consider the chaos of the system to be monotonically correlated to damage value. Chaos can be identified from the effect of the popularity of the products and the store layout. For each products in the shop, the higher the popularity of the product is, the higher the competitive would be resulted in more damage value toward that product. Then we need to consider how damage can be caused from the products , we would assume that the weakness of the product is defined by its size and prize. Damaging more expensive products would obviously cause more damage. Since it's very difficult to calculate how fragile the product is. We will assume the products all have the same fragility. We also assume that the force done by the shoppers are equally distributed. This means that the bigger the product is, the higher probability of it being damaged, and the more force it would get from the shoppers.
\newline

Last but not least, when we consider the layout that influence the damage. We need to assume that when we consider a block of the store containing the products, the shopper can only access the products on the edge, meaning that only the edge of the block can be damaged. So we can calculate the potential damage of each block by measuring the edge or the perimeter if the block can be accessed on all four sides. For the area in front of the shelf, we assume that there's enough space for everyone to walk. In reality, only finite number of people standing in front of the shelf can access the products. However, in our model we would assume that every people in that area can access the products or in other words, can damage the products and we also assume that the messiness for the specific block is affected only by its own messiness and the ones adjacent to it.
\newline
